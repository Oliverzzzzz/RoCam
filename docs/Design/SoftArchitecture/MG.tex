\documentclass[12pt, titlepage]{article}
\pdfinfoomitdate=1
\pdftrailerid{}

\usepackage{fullpage}
\usepackage[round]{natbib}
\usepackage{multirow}
\usepackage{booktabs}
\usepackage{tabularx}
\usepackage{graphicx}
\usepackage{float}
\usepackage{hyperref}
\hypersetup{
    colorlinks,
    citecolor=blue,
    filecolor=black,
    linkcolor=red,
    urlcolor=blue
}

\input{../../Comments}
\input{../../Common}

\newcounter{acnum}
\newcommand{\actheacnum}{AC\theacnum}
\newcommand{\acref}[1]{AC\ref{#1}}

\newcounter{ucnum}
\newcommand{\uctheucnum}{UC\theucnum}
\newcommand{\uref}[1]{UC\ref{#1}}

\newcounter{mnum}
\newcommand{\mthemnum}{M\themnum}
\newcommand{\mref}[1]{M\ref{#1}}

\begin{document}

\title{Module Guide for \progname{}}
\author{\authname}
\date{January 12, 2026}

\maketitle

\pagenumbering{roman}

\section{Revision History}

\begin{tabularx}{\textwidth}{p{3cm}p{2cm}X}
  \toprule {\bf Date} & {\bf Version} & {\bf Notes}              \\
  \midrule
  Nov. 10, 2025       & Rev -1        & Initial Draft            \\
  Nov. 16, 2025       & -             & Fix Peer Review Comments \\
  \bottomrule
\end{tabularx}

\newpage

\section{Reference Material}

This section records information for easy reference.

\subsection{Abbreviations and Acronyms}

\renewcommand{\arraystretch}{1.2}
\begin{tabular}{l l}
  \toprule
  \textbf{symbol} & \textbf{description}                \\
  \midrule
  AC              & Anticipated Change                  \\
  DAG             & Directed Acyclic Graph              \\
  M               & Module                              \\
  MG              & Module Guide                        \\
  OS              & Operating System                    \\
  R               & Requirement                         \\
  SC              & Scientific Computing                \\
  SRS             & Software Requirements Specification \\
  \progname       & Explanation of program name         \\
  UC              & Unlikely Change                     \\
  \wss{etc.}      & \wss{...}                           \\
  \bottomrule
\end{tabular}\\

\newpage

\tableofcontents

\listoftables

\listoffigures

\newpage

\pagenumbering{arabic}

\section{Introduction}

Decomposing a system into modules awa is a commonly accepted approach to
developing software. A module is a work assignment for a programmer or
programming team~\citep{ParnasEtAl1984}. We advocate a decomposition based on
the principle of information hiding~\citep{Parnas1972a}. This principle
supports design for change, because the ``secrets'' that each module hides
represent likely future changes. Design for change is valuable in SC, where
modifications are frequent, especially during initial development as the
solution space is explored.

Our design follows the rules layed out by \citet{ParnasEtAl1984}, as follows:
\begin{itemize}
  \item System details that are likely to change independently should be the secrets of
        separate modules.
  \item Each data structure is implemented in only one module.
  \item Any other program that requires information stored in a module's data
        structures must obtain it by calling access programs belonging to that module.
\end{itemize}

After completing the first stage of the design, the Software Requirements
Specification (SRS), the Module Guide (MG) is developed~\citep{ParnasEtAl1984}.
The MG specifies the modular structure of the system and is intended to allow
both designers and maintainers to easily identify the parts of the software.
The potential readers of this document are as follows:

\begin{itemize}
  \item New project members: This document can be a guide for a new project member to
        easily understand the overall structure and quickly find the relevant modules
        they are searching for.
  \item Maintainers: The hierarchical structure of the module guide improves the
        maintainers' understanding when they need to make changes to the system. It is
        important for a maintainer to update the relevant sections of the document
        after changes have been made.
  \item Designers: Once the module guide has been written, it can be used to check for
        consistency, feasibility, and flexibility. Designers can verify the system in
        various ways, such as consistency among modules, feasibility of the
        decomposition, and flexibility of the design.
\end{itemize}

The rest of the document is organized as follows. Section \ref{SecChange} lists
the anticipated and unlikely changes of the software requirements. Section
\ref{SecMH} summarizes the module decomposition that was constructed according
to the likely changes. Section \ref{SecConnection} specifies the connections
between the software requirements and the modules. Section \ref{SecMD} gives a
detailed description of the modules. Section \ref{SecTM} includes two
traceability matrices. One checks the completeness of the design against the
requirements provided in the SRS. The other shows the relation between
anticipated changes and the modules. Section \ref{SecUse} describes the use
relation between modules.

\section{Anticipated and Unlikely Changes} \label{SecChange}

This section lists possible changes to the system. According to the likeliness
of the change, the possible changes are classified into two categories.
Anticipated changes are listed in Section \ref{SecAchange}, and unlikely
changes are listed in Section \ref{SecUchange}.

\subsection{Anticipated Changes} \label{SecAchange}

Anticipated changes are the source of the information that is to be hidden
inside the modules. Ideally, changing one of the anticipated changes will only
require changing the one module that hides the associated decision. The
approach adapted here is called design for change.

\begin{description}
  \item[\refstepcounter{acnum} \actheacnum \label{acCV}:] The computer vision model. Frequent changes to the vision model might be required to adapt the system to new deployment envirouments.
  \item[\refstepcounter{acnum} \actheacnum \label{acUI}:] The user interface. The user interface may need to be updated based on the feedback of our customers during the field test.
\end{description}

\wss{Anticipated changes relate to changes that would be made in requirements,
  design or implementation choices.  They are not related to changes that are made
  at run-time, like the values of parameters.}

\subsection{Unlikely Changes} \label{SecUchange}

The module design should be as general as possible. However, a general system
is more complex. Sometimes this complexity is not necessary. Fixing some design
decisions at the system architecture stage can simplify the software design. If
these decision should later need to be changed, then many parts of the design
will potentially need to be modified. Hence, it is not intended that these
decisions will be changed.

\begin{description}
  \item[\refstepcounter{ucnum} \uctheucnum \label{ucHW}:]
        The hardware platform. The system relys hardware acceleration for running the
        computer vision model, encoding, and video output. Changing the hardware
        platform may require refactoring all the related algorithms.
  \item[\refstepcounter{ucnum} \uctheucnum \label{ucRocket}:]
        The tracking algorithm. The system is designed to track only rockets. The
        tracking algorithms are designed to only handle the mostly deterministic
        trajectory of a rocket. Changing the tracked object to something other than a
        rocket will make the tracking algorithm less accurate.
  \item[\refstepcounter{ucnum} \uctheucnum \label{ucRemote}:]
        Fundamental changes to the use case. The system is designed to be operated
        remotely by a human operator.
\end{description}

\section{Module Hierarchy} \label{SecMH}

This section provides an overview of the module design. Modules are summarized
in a hierarchy decomposed by secrets in Table \ref{TblMH}. The modules listed
below, which are leaves in the hierarchy tree, are the modules that will
actually be implemented.

\begin{description}
  \item [\refstepcounter{mnum} \mthemnum \label{mJetson}:] Jetson Module
  \item [\refstepcounter{mnum} \mthemnum \label{mCV}:] CV Process Module
  \item [\refstepcounter{mnum} \mthemnum \label{mVideo}:] Live Video Process Module
  \item [\refstepcounter{mnum} \mthemnum \label{mTranscode}:] Transcode Process Module
  \item [\refstepcounter{mnum} \mthemnum \label{mStream}:] Video Stream Abstraction Module
  \item [\refstepcounter{mnum} \mthemnum \label{mControl}:] Control Process Module
  \item [\refstepcounter{mnum} \mthemnum \label{mCVManagement}:] CV Process Management Module
  \item [\refstepcounter{mnum} \mthemnum \label{mVideoManagement}:] Live Video Process Management Module
  \item [\refstepcounter{mnum} \mthemnum \label{mTranscodeManagement}:] Transcode Process Management Module
  \item [\refstepcounter{mnum} \mthemnum \label{mTrack}:] Tracking Module
  \item [\refstepcounter{mnum} \mthemnum \label{mGimbal}:] Gimbal Abstraction Module
  \item [\refstepcounter{mnum} \mthemnum \label{mSerial}:] Serial Abstraction Module
  \item [\refstepcounter{mnum} \mthemnum \label{mStatus}:] System Status Module
  \item [\refstepcounter{mnum} \mthemnum \label{mRecord}:] Recording Database Module
  \item [\refstepcounter{mnum} \mthemnum \label{mState}:] State Management Module
  \item [\refstepcounter{mnum} \mthemnum \label{mAPI}:] API Gateway Module
  \item [\refstepcounter{mnum} \mthemnum \label{mUI}:] UI Module
  \item [\refstepcounter{mnum} \mthemnum \label{mPreview}:] Preview Module
  \item [\refstepcounter{mnum} \mthemnum \label{mManual}:] Manual Control Module
  \item [\refstepcounter{mnum} \mthemnum \label{mRecordMng}:] Recording Management Module
  \item [\refstepcounter{mnum} \mthemnum \label{mConfig}:] Configuration Module
\end{description}

\begin{table}[h!]
  \centering
  \begin{tabular}{p{0.3\textwidth} p{0.6\textwidth}}
    \toprule
    \textbf{Level 1}               & \textbf{Level 2}                                                  \\
    \midrule

    Jetson Module (\mref{mJetson}) & CV Process Module (\mref{mCV})                                    \\  & CV Process
    Management Module (\mref{mCVManagement})                                                           \\  & Live Video Process Module
    (\mref{mVideo})                                                                                    \\  & Live Video Process Management Module
    (\mref{mVideoManagement})                                                                          \\ & Transcode Process Module (\mref{mTranscode}) \\
                                   & Transcode Process Management Module (\mref{mTranscodeManagement}) \\  &
    Tracking Module (\mref{mTrack})                                                                    \\  & Gimbal Abstraction Module (\mref{mGimbal})
    \\ & Serial Abstraction Module (\mref{mSerial}) \\  & System Status Module
    (\mref{mStatus})                                                                                   \\ & Recording Database Module (\mref{mRecord}) \\  & State
    Management Module (\mref{mState})                                                                  \\  & Video Stream Abstraction Module
    (\mref{mStream})                                                                                   \\ & API Gateway Module (\mref{mAPI}) \\

    \midrule

    UI Module (\mref{mUI})         & Preview Module (\mref{mPreview})                                  \\  & Manual Control
    Module (\mref{mManual})                                                                            \\ & Recording Management Module (\mref{mRecordMng}) \\
                                   & Configuration Module (\mref{mConfig})                             \\ \bottomrule

  \end{tabular}
  \caption{Module Hierarchy}
  \label{TblMH}
\end{table}

\section{Connection Between Requirements and Design} \label{SecConnection}

The design of the system is intended to satisfy the requirements developed in
the SRS. In this stage, the system is decomposed into modules. The connection
between requirements and modules is listed in Table~\ref{TblRT}.

\wss{The intention of this section is to document decisions that are made
  ``between'' the requirements and the design.  To satisfy some requirements,
  design decisions need to be made.  Rather than make these decisions implicit,
  they are explicitly recorded here.  For instance, if a program has security
  requirements, a specific design decision may be made to satisfy those
  requirements with a password.}

\section{Module Decomposition} \label{SecMD}

Modules are decomposed according to the principle of ``information hiding''
proposed by \citet{ParnasEtAl1984}. The \emph{Secrets} field in a module
decomposition is a brief statement of the design decision hidden by the module.
The \emph{Services} field specifies \emph{what} the module will do without
documenting \emph{how} to do it. For each module, a suggestion for the
implementing software is given under the \emph{Implemented By} title. If the
entry is \emph{OS}, this means that the module is provided by the operating
system or by standard programming language libraries. \emph{\progname{}} means
the module will be implemented by the \progname{} software.

Only the leaf modules in the hierarchy have to be implemented. If a dash
(\emph{--}) is shown,this means that the module is not a leaf and will not have
to be implemented.

\subsection{Jetson Module (\mref{mJetson})}

The jetson module contains all the code that runs on the Nvidia Jetson.

\subsection{CV Process Module (\mref{mCV})}

\begin{description}
  \item[Secrets:] \leavevmode
    \begin{itemize}
      \item How to retreive and process the video feed from camera
      \item Computer vision algorithms
      \item How to start/stop recording
    \end{itemize}
  \item[Services:]  \leavevmode
    \begin{itemize}
      \item Detects rocket location from the camera feed
      \item Record camera feed to disk
      \item Generate preview video feed
      \item Generate live stream video feed
    \end{itemize}
  \item[Implemented By:] \progname
  \item[Type of Module:] Process
\end{description}

\subsection{Live Video Process Module (\mref{mVideo})}

\begin{description}
  \item[Secrets:] \leavevmode
    \begin{itemize}
      \item How to stablize and overlay text to the live video feed
      \item How to handle CV process crashes
      \item How to output video to HDMI
    \end{itemize}
  \item[Services:]  \leavevmode
    \begin{itemize}
      \item Stablize and overlay text to the live stream video feed
      \item Output the live stream video feed to HDMI
    \end{itemize}
  \item[Implemented By:] \progname
  \item[Type of Module:] Process
\end{description}

\subsection{Transcode Process Module (\mref{mTranscode})}

\begin{description}
  \item[Secrets:] \leavevmode
    \begin{itemize}
      \item How to encode recorded video to a more efficient format
      \item How to stablize and overlay text to the recorded video
    \end{itemize}
  \item[Services:] Transcode a recording with optional stablization and text overlay
  \item[Implemented By:] \progname
  \item[Type of Module:] Process
\end{description}

\subsection{Video Stream Abstraction Module (\mref{mStream})}

\begin{description}
  \item[Secrets:] How to efficiently handle video stream data
  \item[Services:] Video stream retrieval, routing, processing, and output.
  \item[Implemented By:] Nvidia Deepstream SDK
  \item[Type of Module:] Library
\end{description}

\subsection{Control Process Module (\mref{mControl})}

The control process module orchestrates all the processes and manages the system state.

\subsection{CV Process Management Module (\mref{mCVManagement})}

\begin{description}
  \item[Secrets:] \leavevmode
    \begin{itemize}
      \item How to start the CV process
      \item How to communicate with the CV process
    \end{itemize}
  \item[Services:] Allows the Control Process Module (\mref{mControl}) to access service provided by the CV Process Module (\mref{mCV})
  \item[Implemented By:] \progname
  \item[Type of Module:] Class
\end{description}

\subsection{Live Video Process Management Module (\mref{mVideoManagement})}

\begin{description}
  \item[Secrets:] \leavevmode
    \begin{itemize}
      \item How to start the Live Video process
      \item How to communicate with the Live Video process
    \end{itemize}
  \item[Services:] Allows the Control Process Module (\mref{mControl}) to access service provided by the Live Video Process Module (\mref{mVideo})
  \item[Implemented By:] \progname
  \item[Type of Module:] Class
\end{description}

\subsection{Transcode Process Management Module (\mref{mTranscodeManagement})}

\begin{description}
  \item[Secrets:] \leavevmode
    \begin{itemize}
      \item How to start the Transcode process
      \item How to communicate with the Transcode process
    \end{itemize}
  \item[Services:] Allows the Control Process Module (\mref{mControl}) to access service provided by the Transcode Process Module (\mref{mTranscode})
  \item[Implemented By:] \progname
  \item[Type of Module:] Class
\end{description}

\subsection{Tracking Module (\mref{mTrack})}

\begin{description}
  \item[Secrets:] The tracking algorithm
  \item[Services:] Calculates the angle of the gimbal required to keep the rocket in the frame.
  \item[Implemented By:] \progname
  \item[Type of Module:] Class
\end{description}

\subsection{Gimbal Abstraction Module (\mref{mGimbal})}

\begin{description}
  \item[Secrets:] The communication protocol of the gimbal
  \item[Services:] \leavevmode
    \begin{itemize}
      \item Command the gimbal to go to a specified angle
      \item Measure the current angle of the gimbal
    \end{itemize}
  \item[Implemented By:] \progname
  \item[Type of Module:] Class
\end{description}

\subsection{Serial Abstraction Module (\mref{mSerial})}

\begin{description}
  \item[Secrets:] How to interface with the hardware serial ports.
  \item[Services:] Send or receive bytes through the serial port of the hardware.
  \item[Implemented By:] pip package pyserial
  \item[Type of Module:] Library
\end{description}

\subsection{System Status Module (\mref{mStatus})}

\begin{description}
  \item[Secrets:] How to retreive various system status
  \item[Services:] Provide various system status (e.g. GPU ultilization, core temperature)
  \item[Implemented By:] pip package jetson-stats
  \item[Type of Module:] Library
\end{description}

\subsection{Recording Database Module (\mref{mRecord})}

\begin{description}
  \item[Secrets:] How to track all the recordings and logs
  \item[Services:] Track all the files that store recordings and logs
  \item[Implemented By:] \progname
  \item[Type of Module:] Class
\end{description}

\subsection{State Management Module (\mref{mState})}

\begin{description}
  \item[Secrets:] \leavevmode
    \begin{itemize}
      \item The data structure to store the state of the system
      \item State transition logic
    \end{itemize}
  \item[Services:] \leavevmode
    \begin{itemize}
      \item Manages the state transitions of the system
      \item Aggregate data for the API Gateway
    \end{itemize}
  \item[Implemented By:] \progname
  \item[Type of Module:] Class
\end{description}

\subsection{API Gateway Module (\mref{mAPI})}

\begin{description}
  \item[Secrets:] The API endpoints and their corresponding logic.
  \item[Services:] Provides the API endpoints for the UI.
  \item[Implemented By:] \progname
  \item[Type of Module:] Class
\end{description}

\subsection{UI Module (\mref{mUI})}

The UI module contains all the code that the operator uses to interact with the
system via the web interface.

\subsection{Preview Module (\mref{mPreview})}

\begin{description}
  \item[Secrets:] How to retrieve and display the video feed from the API Gateway.
  \item[Services:] Displays preview of the video feed to the user.
  \item[Implemented By:] \progname
  \item[Type of Module:] Library
\end{description}

\subsection{Manual Control Module (\mref{mManual})}

\begin{description}
  \item[Secrets:] Intuitive user interface for manual control, and how to send manual control commands to the API Gateway.
  \item[Services:] Allows the operator to control the gimbal manually via a user interface.
  \item[Implemented By:] \progname
  \item[Type of Module:] Library
\end{description}

\subsection{Recording Management Module (\mref{mRecordMng})}

\begin{description}
  \item[Secrets:] How to list, delete, and download the recorded videos and log files.
  \item[Services:] Allows the operator to list, delete, and download the recorded videos and log files.
  \item[Implemented By:] \progname
  \item[Type of Module:] Library
\end{description}

\subsection{Configuration Module (\mref{mConfig})}

\begin{description}
  \item[Secrets:] The UI for configuring the system.
  \item[Services:] Allows the operator to configure the system via a user interface.
  \item[Implemented By:] \progname
  \item[Type of Module:] Library
\end{description}

\section{Traceability Matrix} \label{SecTM}

This section shows two traceability matrices: between the modules and the
requirements and between the modules and the anticipated changes.

% the table should use mref, the requirements should be named, use something

\begin{table}[H]
  \centering
  \begin{tabular}{p{0.2\textwidth} p{0.6\textwidth}}
    \toprule
    \textbf{Req.} & \textbf{Modules}                                                                           \\
    \midrule
    FR-1          & \mref{mJetson}, \mref{mVideo}, \mref{mCV}                                                  \\
    FR-2          & \mref{mGimbal}, \mref{mTrack}, \mref{mManual}, \mref{mAPI}                                 \\
    FR-3          & \mref{mCV}, \mref{mJetson}                                                                 \\
    FR-4          & \mref{mState}, \mref{mGimbal}, \mref{mCV}                                                  \\
    FR-5          & \mref{mState}, \mref{mCV}, \mref{mTrack}                                                   \\
    FR-6          & \mref{mManual}, \mref{mAPI}, \mref{mGimbal}, \mref{mState}                                 \\
    FR-7          & \mref{mTrack}, \mref{mCV}, \mref{mGimbal}, \mref{mState}                                   \\
    FR-8          & \mref{mVideo}, \mref{mJetson}, \mref{mAPI}, \mref{mUI}, \mref{mPreview}                    \\
    FR-9          & \mref{mVideo}, \mref{mGimbal}, \mref{mUI}, \mref{mAPI}, \mref{mPreview}                    \\
    FR-10         & \mref{mPreview}, \mref{mUI}, \mref{mState}, \mref{mAPI}, \mref{mVideo}                     \\
    FR-11         & \mref{mRecord}, \mref{mRecordMng}, \mref{mAPI}, \mref{mUI}, \mref{mPreview}, \mref{mVideo} \\
    FR-12         & \mref{mRecordMng}, \mref{mAPI}, \mref{mRecord}, \mref{mUI},                                \\
    \bottomrule
  \end{tabular}
  \caption{Trace Between Functional Requirements and Modules}
  \label{TblRT}
\end{table}

% M1: Jetson Module (\mref{mJetson})
% M2: Gimbal Abstraction Module (\mref{mGimbal})
% M3: Computer Vision Module (\mref{mCV})
% M4: Tracking Module (\mref{mTrack})
% M5: Output Video Module (\mref{mVideo}) 
% M6: Recording Module (\mref{mRecord})
% M7: State Management Module (\mref{mState})
% M8: API Gateway Module (\mref{mAPI})
% M9: UI Module (\mref{mUI})
% M10: Preview Module (\mref{mPreview})
% M11: Manual Control Module (\mref{mManual})
% M12: Recording Management Module (\mref{mRecordMng})
% M13: Configuration Module (\mref{mConfig})

\begin{table}[H]
  \centering
  \begin{tabular}{p{0.15\textwidth} p{0.75\textwidth}}
    \toprule
    \textbf{Req.} & \textbf{Modules}                               \\
    \midrule
    AR-1          & \mref{mUI},  \mref{mRecordMng}, \mref{mConfig} \\
    AR-2          & \mref{mUI},  \mref{mRecordMng}                 \\
    SR-1          & \mref{mUI}, \mref{mConfig}                     \\
    \bottomrule
  \end{tabular}
  \caption{Trace Between Appearance and Style Requirements and Modules}
  \label{TblAppearanceStyleModules}
\end{table}

\begin{table}[H]
  \centering
  \begin{tabular}{p{0.15\textwidth} p{0.75\textwidth}}
    \toprule
    \textbf{Req.} & \textbf{Modules}                                               \\
    \midrule
    EZ-1          & \mref{mUI}                                                     \\
    EZ-2          & \mref{mUI}, \mref{mAPI}                                        \\
    EZ-3          & \mref{mUI},  \mref{mAPI}, \mref{mConfig}                       \\
    EZ-4          & \mref{mUI},  \mref{mConfig}, \mref{mAPI}, \mref{mState}        \\
    PI-1          & \mref{mUI}, \mref{mConfig}, \mref{mPreview}, \mref{mRecordMng} \\
    PI-2          & \mref{mUI}, \mref{mConfig}                                     \\
    LR-1          & \mref{mUI}                                                     \\
    UPR-1         & \mref{mUI}                                                     \\
    UPR-2         & \mref{mUI}, \mref{mAPI}                                        \\
    UPR-3         & \mref{mUI}                                                     \\
    AR-1          & \mref{mUI},  \mref{mState}, \mref{mAPI}                        \\
    \bottomrule
  \end{tabular}
  \caption{Trace Usability and Humanity Requirements and Modules}
  \label{TblUsabilityModules}
\end{table}

\begin{table}[H]
  \centering
  \begin{tabular}{p{0.18\textwidth} p{0.72\textwidth}}
    \toprule
    \textbf{Req.} & \textbf{Modules}                                                                      \\
    \midrule
    SLR-1         & \mref{mJetson}, \mref{mCV}, \mref{mTrack}, \mref{mGimbal}, \mref{mState}              \\
    SLR-2         & \mref{mJetson}, \mref{mCV}, \mref{mVideo}, \mref{mTrack}                              \\
    SLR-3         & \mref{mVideo}, \mref{mJetson}, \mref{mAPI}                                            \\
    SCR-1         & \mref{mUI}, \mref{mManual}, \mref{mState}, \mref{mAPI}, \mref{mGimbal}                \\
    SCR-2         & \mref{mTrack}, \mref{mState}, \mref{mCV}, \mref{mGimbal}, \mref{mJetson}              \\
    SCR-3         & \mref{mUI}, \mref{mAPI}, \mref{mState}, \mref{mJetson}                                \\
    SCR-4         & \mref{mUI}, \mref{mState}, \mref{mAPI}, \mref{mGimbal}                                \\
    PAR-1         & \mref{mTrack}, \mref{mCV}, \mref{mGimbal}, \mref{mVideo}, \mref{mAPI}, \mref{mJetson} \\
    RFR-1         & \mref{mUI}, \mref{mAPI}, \mref{mState}                                                \\
    RFR-2         & \mref{mState}, \mref{mTrack}, \mref{mGimbal}, \mref{mAPI}                             \\
    RFR-3         & \mref{mJetson}, \mref{mAPI}, \mref{mUI}, \mref{mState}, \mref{mJetson}                \\
    RFR-4         & \mref{mTrack}, \mref{mCV}, \mref{mState}, \mref{mGimbal}, \mref{mJetson}              \\
    RFR-5         & \mref{mTrack}, \mref{mCV}, \mref{mGimbal}, \mref{mState}                              \\
    CR-1          & \mref{mRecord}, \mref{mRecordMng}, \mref{mJetson}                                     \\
    \bottomrule
  \end{tabular}
  \caption{Trace Between Performance Requirements and Modules}
  \label{TblPerformanceModules}
\end{table}

\begin{table}[H]
  \centering
  \begin{tabular}{p{0.18\textwidth} p{0.72\textwidth}}
    \toprule
    \textbf{Req.} & \textbf{Modules}                                  \\
    \midrule
    EPE-1         & \mref{mJetson}, \mref{mGimbal}                    \\
    INT-1         & \mref{mVideo}, \mref{mJetson}                     \\
    INT-2         & \mref{mGimbal}, \mref{mAPI}, \mref{mJetson}       \\
    RR-1          & \mref{mRecord}, \mref{mRecordMng}, \mref{mJetson} \\
    \bottomrule
  \end{tabular}
  \caption{Trace Between Operational and Environmental Requirements and Modules}
  \label{TblOperationalModules}
\end{table}

\begin{table}[H]
  \centering
  \begin{tabular}{p{0.18\textwidth} p{0.72\textwidth}}
    \toprule
    \textbf{Req.} & \textbf{Modules}                                                                                      \\
    \midrule
    IR-1          & \mref{mUI}, \mref{mManual}, \mref{mAPI}, \mref{mState}, \mref{mGimbal}, \mref{mJetson}, \mref{mTrack} \\
    AUR-1         & \mref{mRecord}, \mref{mRecordMng}, \mref{mAPI}, \mref{mState}, \mref{mUI}                             \\
    \bottomrule
  \end{tabular}
  \caption{Trace Between Security Requirements and Modules}
  \label{TblSecurityModules}
\end{table}

\begin{table}[H]
  \centering
  \begin{tabular}{p{0.18\textwidth} p{0.72\textwidth}}
    \toprule
    \textbf{Req.} & \textbf{Modules}                            \\
    \midrule
    CR-1          & \mref{mUI}, \mref{mPreview}, \mref{mConfig} \\
    CR-2          & \mref{mUI}, \mref{mPreview}, \mref{mConfig} \\
    \bottomrule
  \end{tabular}
  \caption{Trace Between Cultural Requirements and Modules}
  \label{TblCulturalModules}
\end{table}

\begin{table}[H]
  \centering
  \begin{tabular}{p{0.18\textwidth} p{0.72\textwidth}}
    \toprule
    \textbf{Req.} & \textbf{Modules}                                                           \\
    \midrule
    LR-1          & \mref{mUI}, \mref{mRecord}, \mref{mRecordMng}, \mref{mAPI}, \mref{mJetson} \\
    \bottomrule
  \end{tabular}
  \caption{Trace Between Compliance Requirements and Modules}
  \label{TblComplianceModules}
\end{table}

\begin{table}[H]
  \centering
  \begin{tabular}{p{0.25\textwidth} p{0.65\textwidth}}
    \toprule
    \textbf{Anticipated Change} & \textbf{Modules}                                            \\
    \midrule
    \acref{acCV}                & \mref{mCV}, \mref{mJetson}                                  \\
    \acref{acUI}                & \mref{mUI}, \mref{mPreview}, \mref{mManual}, \mref{mConfig} \\
    \bottomrule
  \end{tabular}
  \caption{Trace Between Anticipated Changes and Modules}
  \label{TblACModules}
\end{table}

\begin{table}[H]
  \centering
  \begin{tabular}{p{0.25\textwidth} p{0.65\textwidth}}
    \toprule
    \textbf{Unlikely Change} & \textbf{Modules}                                               \\
    \midrule
    \uref{ucHW}              & \mref{mJetson}, \mref{mGimbal}, \mref{mSerial}, \mref{mStream} \\
    \uref{ucRocket}          & \mref{mTrack}, \mref{mCV}                                      \\
    \uref{ucRemote}          & \mref{mUI}, \mref{mAPI}, \mref{mManual}                        \\
    \bottomrule
  \end{tabular}
  \caption{Trace Between Unlikely Changes and Modules}
  \label{TblUCModules}
\end{table}

\section{Use Hierarchy Between Modules} \label{SecUse}

In this section, the uses hierarchy between modules is provided.
\citet{Parnas1978} said of two programs A and B that A {\em uses} B if correct
execution of B may be necessary for A to complete the task described in its
specification. That is, A {\em uses} B if there exist situations in which the
correct functioning of A depends upon the availability of a correct
implementation of B. Figure \ref{FigUH} illustrates the use relation between
the modules. It can be seen that the graph is a directed acyclic graph (DAG).
Each level of the hierarchy offers a testable and usable subset of the system,
and modules in the higher level of the hierarchy are essentially simpler
because they use modules from the lower levels.

\wss{The uses relation is not a data flow diagram.  In the code there will often
  be an import statement in module A when it directly uses module B.  Module B
  provides the services that module A needs.  The code for module A needs to be
  able to see these services (hence the import statement).  Since the uses
  relation is transitive, there is a use relation without an import, but the
  arrows in the diagram typically correspond to the presence of import statement.}

\wss{If module A uses module B, the arrow is directed from A to B.}

\begin{figure}[H]
  \centering
  \includegraphics[width=\textwidth,height=\textheight,keepaspectratio]{../../Images/module_use_hierarchy.png}
  \caption{Use hierarchy among modules}
  \label{FigUH}
\end{figure}

%\section*{References}

\section{User Interfaces}

The system provides a single-page web interface for the operator. This web UI
is the only entry point for interacting with the system, including live
preview, state switching, manual gimbal control, and recording management.

\subsection{Main Control Page}

Figure~\ref{fig:ui_overview} shows the main \textit{Control} page. The left
panel displays the live camera feed for situational awareness. The right panel
displays the current system state (e.g., \textit{Disarmed} or \textit{Armed})
and real-time gimbal pose telemetry, including \textit{Tilt} and \textit{Pan}
angles. The top-right corner includes \textit{Fullscreen} for better field
visibility and an \textit{Emergency Stop} control for safety-critical shutdown.

% ---- Image 1: overview screenshot ----
\begin{figure}[H]
  \centering
  \includegraphics[width=\textwidth]{../../Images/rocam_ui_overview.png}
  \caption{RoCam web interface overview (Control page).}
  \label{fig:ui_overview}
\end{figure}

\subsection{Disarmed Mode: Manual Gimbal Control}

In \textit{Disarmed} mode, the operator can manually control the gimbal using
the on-screen directional pad (up/down/left/right). This mode is used for setup
and calibration (e.g., aiming at the expected launch area before liftoff) and
as a fallback when manual intervention is required. Manual control commands are
sent to the backend through the API Gateway, which relays them to the gimbal
control logic running on the Jetson. The UI continuously updates the displayed
gimbal angles so the emphasizes to the operator the immediate effect of each
command.

% ---- Image 2: disarmed/manual control screenshot ----
\begin{figure}[H]
  \centering
  \includegraphics[width=\textwidth]{../../Images/rocam_ui_disarmed.png}
  \caption{Disarmed mode with manual gimbal controls and real-time tilt/pan feedback.}
  \label{fig:ui_disarmed}
\end{figure}

\subsection{Armed Mode: Automatic Tracking}

In \textit{Armed} mode, the system runs the rocket detection and tracking
pipeline. The Computer Vision and Tracking modules estimate the target location
in the frame and compute the gimbal control commands to keep the rocket
centered. The operator monitors the live preview and can transition to
\textit{Disarmed} mode to take manual control if needed.

\subsection{Recordings}

The UI also provides a \textit{Recordings} section for managing recorded videos
(and associated logs/metadata). The operator can start/stop recordings during
operation and later list, download, or delete recordings through the Recording
Management functions exposed by the backend API.

\section{Design of Communication Protocols}

\subsection{Gimbal Protocol}

The Gimbal Protocol is specified at:
\href{https://github.com/SpaceY-Labs/RoCam/blob/main/docs/Design/GimbalProtocol.pdf}{https://github.com/SpaceY-Labs/RoCam/blob/main/docs/Design/GimbalProtocol.pdf}

\wss{If appropriate}

\section{Timeline}

\wss{Schedule of tasks and who is responsible}

\wss{You can point to GitHub if this information is included there}

For timeline management, we use GitHub Projects. The timeline for implementing
each of the modules can be found at
\href{https://github.com/orgs/SpaceY-Labs/projects/2}{https://github.com/orgs/SpaceY-Labs/projects/2}.

\bibliographystyle {plainnat}
\bibliography{../../../refs/References}

\newpage{}

\end{document}
