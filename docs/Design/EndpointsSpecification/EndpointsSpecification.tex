\documentclass[12pt]{article}

\usepackage{graphicx}
\usepackage{float}
\usepackage{hyperref}
\usepackage{tabularx}
\usepackage{booktabs}
\usepackage[margin=1in]{geometry}

\title{Endpoints Specification (Swagger/OpenAPI)}
\author{Zifan Si}
\date{January 14, 2026}

\begin{document}
\maketitle
\tableofcontents
\newpage

\section{REST API Communication Protocol (OpenAPI)}
This project provides an HTTP REST API for controlling and monitoring the
gimbal system. The API contract is documented using an OpenAPI specification
and is viewable in Swagger Editor (\url{https://editor.swagger.io/}). The API
is used by the front-end/controller to issue commands and query current system
state.

\subsection{Transport and Data Format}
\begin{itemize}
  \item \textbf{Transport:} HTTP
  \item \textbf{Style:} REST-style endpoints under \texttt{/api/*}
  \item \textbf{Content Type:} \texttt{application/json}
  \item \textbf{Encoding:} JSON request/response bodies
\end{itemize}

\subsection{Servers / Deployment Targets}
Swagger UI provides local server presets for development:
\begin{itemize}
  \item \textbf{Frontend development default:} \texttt{http://localhost:5000}
  \item \textbf{Flask default (port 80):} \texttt{http://localhost}
\end{itemize}

\subsection{Communication Model}
The REST API follows a \textbf{request-response} model:
\begin{enumerate}
  \item The client initiates requests to the back-end service.
  \item The server responds with an HTTP status code and a JSON response body.
  \item Action endpoints (e.g., arm/disarm) may return an empty response body and rely
        on the HTTP status code to indicate success.
\end{enumerate}

\subsection{Endpoint Reference}
Table~\ref{tab:rest-endpoints} summarizes the endpoints.

\begin{table}[H]
  \centering
  \begin{tabularx}{\textwidth}{l l X}
    \toprule
    \textbf{Method} & \textbf{Path}                  & \textbf{Purpose}                                       \\
    \midrule
    POST            & \texttt{/api/status}           & Get current system status.                             \\
    POST            & \texttt{/api/manual\_move}     & Move the gimbal in a direction (relative control).     \\
    POST            & \texttt{/api/manual\_move\_to} & Move the gimbal to absolute angles (pan/tilt targets). \\
    POST            & \texttt{/api/arm}              & Arm tracking (enable tracking loop).                   \\
    POST            & \texttt{/api/disarm}           & Disarm tracking (disable tracking loop).               \\
    \bottomrule
  \end{tabularx}
  \caption{REST API endpoints (Swagger/OpenAPI documented).}
  \label{tab:rest-endpoints}
\end{table}

\subsection{Schemas (JSON Data Types)}
Swagger UI defines the following schemas.

\subsubsection{StatusResponse}
Returned by \texttt{POST /api/status}.
\begin{itemize}
  \item \texttt{armed} (required): tracking armed state
  \item \texttt{tilt} (required): current tilt angle
  \item \texttt{pan} (required): current pan angle
  \item \texttt{preview} (required): preview-related state
  \item \texttt{bbox} (nullable): optional bounding box of detected target
\end{itemize}

\subsubsection{BoundingBox (nullable)}
\begin{itemize}
  \item \texttt{pts\_s} (required): points/coordinate representation
  \item \texttt{conf} (required): confidence score
  \item \texttt{left} (required): left coordinate
  \item \texttt{top} (required): top coordinate
  \item \texttt{width} (required): width
  \item \texttt{height} (required): height
\end{itemize}

\subsubsection{ManualMoveRequest}
Request body for \texttt{POST /api/manual\_move}.
\begin{itemize}
  \item \texttt{direction} (required): direction command (e.g., up/down/left/right)
\end{itemize}

\subsubsection{ManualMoveToRequest}
Request body for \texttt{POST /api/manual\_move\_to}.
\begin{itemize}
  \item \texttt{tilt} (required): target tilt angle
  \item \texttt{pan} (required): target pan angle
\end{itemize}

\subsubsection{EmptyResponse}
Used for action endpoints (e.g., \texttt{/api/arm}, \texttt{/api/disarm}) where
a success status code indicates completion and no response fields are required.

\subsection{Example Request/Response Flows}
\paragraph{Get Status}
\begin{verbatim}
POST /api/status
Content-Type: application/json
\end{verbatim}

\paragraph{Manual Move (Directional)}
\begin{verbatim}
POST /api/manual_move
Content-Type: application/json

{"direction":"left"}
\end{verbatim}

\paragraph{Manual Move (Absolute Angles)}
\begin{verbatim}
POST /api/manual_move_to
Content-Type: application/json

{"tilt":10.0,"pan":-5.0}
\end{verbatim}

\paragraph{Arm / Disarm Tracking}
\begin{verbatim}
POST /api/arm
POST /api/disarm
\end{verbatim}

\end{document}
